\textbf{Name:} \\

\medskip

\textbf{Conspirators:} 

\medskip
\medskip

\hrule

\medskip


\begin{enumerate}

\item

  An amuse-bouche for scaling, to signal the flavors ahead:

  Examine current weight lifting records for
  the snatch, clean and jerk, and the total for scaling
  with body mass (three regressions).
  Do so for both women and men's records.

  For weight classes, take the upper limit 
  for the mass of the lifter.

  Wikipedia is an excellent source.

  \begin{enumerate}
  \item 
    How well does 2/3 scaling hold up?
  \item 
    Normalized by the scaling you determine,
    who holds the overall, rescaled world record?

    Normalization here means relative:
    $$
    100 \times 
    \left(
    \frac{
      M_{\rm world record}
    }{
      c M_{\rm weight class}^{\beta}.
    }
    -1
    \right),
    $$
    where $c$ and $\beta$ are the parameters
    determined from a linear fit.

  \end{enumerate}

  
   \solutionstart

   %% solution goes here

   \solutionend

\item
  Some kitchen table preparation for for power-law size distributions:
  
  Consider a random variable $X$ with 
  a probability distribution given by
  $$
  P(x) = c x^{-\gamma}
  $$
  where $c$ is a normalization constant,
  and $0 < a \le x \le b $.
  ($a$ and $b$ are the lower and upper cutoffs respectively.)
  A Perishing Monk tells you to assume that $\gamma > 1$, that $a > 0$ always, and allow
  for the possibility that $b \rightarrow \infty$.
  And then the Monk disappears.

  \begin{enumerate}

  \item
    Determine $c$.  

    
   \solutionstart

   %% solution goes here

   \solutionend

  \item
    Why did the Perishing Monk tell us to assume $\gamma>1$?

    Think about what happens as $b \rightarrow \infty$.
    
    
   \solutionstart

   %% solution goes here

   \solutionend

  \end{enumerate}

  
\end{enumerate}

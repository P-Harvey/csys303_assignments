\textbf{Name:} \\

\medskip

\textbf{Conspirators:} 

\medskip
\medskip

\hrule

\medskip


%% \assignmentsonly{\pleasesubmitprojectdraft}
%%
%% \medskip
%% \hrule

For Q1--5,
you'll further explore the Google data set
you examined earlier.

Q6 prepares for allotaxonometry.

%% For Q6--8, you'll explore random walks,
%% touching the Central Limit Theorem and
%% the first return problem.
%% There and back again.


%% For Q6--7, you'll examine how the largest
%$ sample size grows with the number of samples.


\begin{enumerate}


\item 
  Plot the complementary cumulative distribution function
  (CCDF).

%%  \cliplink{06f}

  
   \solutionstart

   %% solution goes here

   \solutionend

\item 
  Using standard linear regression,
  measure the exponent $\gamma-1$
  where $\gamma$ is the exponent of
  the underlying distribution function.
  Identify and use a range
  of frequencies for which scaling
  appears consistent.
  Report the 95\% confidence interval
  for your estimate.

  You will find two scaling regimes---please examine them both.

  
   \solutionstart

   %% solution goes here

   \solutionend

\item
  Size-rank plots:
  
  Using the alternate data set providing the raw word frequencies,
  plot word frequency as a function of rank
  in the manner of Zipf.

  \textbf{Hint:} you will not be able to plot all points
  (there are close to 14 million)
  so think about how to plot a subsample that still
  shows the full form.
  
  
   \solutionstart

   %% solution goes here

   \solutionend

\item 
  Using
  standard linear regression,
  measure $\alpha$, Zipf's exponent.
  Report the 95\% confidence interval
  for your estimate.

  Again, you will find two regimes.

  
   \solutionstart

   %% solution goes here

   \solutionend

\item 
  For each scaling regime, write down how $\gamma$ and $\alpha$ are related (per lectures)
  and check how this expression works
  for your estimates here.

  
   \solutionstart

   %% solution goes here

   \solutionend


 
\item (3 + 3) 
  \textbf{Baby name frequencies in the US:}

  Note: We will use this data set again in the next assignment.

  \begin{enumerate}
  \item 
    Plot the Complementary Cumulative Frequency Distributions and Size-rank plots (Zipf's law)
    for the following:

    \begin{enumerate}
    \item
      Baby girl names in 1952.
    \item
      Baby boy names in 1952.
    \item
      Baby girl names in 2002.
    \item
      Baby boy names in 2002.
    \end{enumerate}

    Note that you will have counts that will make the Zipf distribution
    easy to plot straight away.

    From these counts,
    you will have to create
    the distributions 
    $N_{k}$ and $N_{\ge k}$.

  \item

    As you did for the Google data set, fit regression lines
    and report values of
    $\gamma$
    and
    the Zipf exponent $\alpha$.

    BUT: Only fit lines if fitting lines make sense!

    You may only have one region of scaling or zero.

  \end{enumerate}

  We will revisit these distributions in following assignments.

  \textbf{Download:}

  Data for 1880 through 2018:

  \wordwikilink{http://pdodds.w3.uvm.edu/permanent-share/pocs-babynames.zip}{http://pdodds.w3.uvm.edu/permanent-share/pocs-babynames.zip} (8.0M)

  \textbf{Files:}
  
  For each year, Zipf distribution of counts are stored in:
  \texttt{names-girlsYYYY.txt}
  and 
  \texttt{names-boyYYYY.txt}.

  For normalization to estimate rates,
  total number of births per year: \texttt{births\_per\_year.txt}.
  For this question, you do not need to determine rates, and this
  file is included for completeness.

  For privacy, names with less than 5 counts are excluded.

  The rare are legion and, for baby names, hidden.

  \textbf{Notes:}

  You should be able to re-use scripts from previous assignments.

  Data is based on names registered through Social Security within the US.

  \textbf{Source:}

  Baby name dataset available here:\\
  \wordwikilink{https://catalog.data.gov/dataset?tags=baby-names}{https://catalog.data.gov/dataset?tags=baby-names}.
  Separate dataset for total births available here:\\
  \wordwikilink{https://www.ssa.gov/oact/babynames/numberUSbirths.html}{https://ssa.gov/oact/babynames/numberUSbirths.html}.

  
   \solutionstart

   %% solution goes here

   \solutionend

\item
  
  Install Matlab on your machine of choice.

  We'll use Matlab in the next assignment.

  3 points for just getting this done and reporting faithfully that you did succeed.

  If you already have Matlab, then this is a freebie.

  For what we're going to do, there will be some roadblocks if you are not using a UNIX machine.

  Linux machines will work, Apple/Mac will work (because Mac OS is UNIX underneath),
  but Windows will present some problems.

  Note: For future happiness, we encourage students to use Python in general. Maybe R.
  Julia too.  We're using a complicated piece of machinery that only exists fully
  realized in Matlab.

\end{enumerate}


\textbf{Name:} \\

\medskip

\textbf{Conspirators:} 

\medskip
\medskip

\hrule

\medskip


%% For 2023:
%% Could clean up the central limit theorem calc
%% Penguins works on the line before


%% HELOOOOOO
%% Move power law questions forward ...

%% add question on matrices and random walks

%% thanks. a few comments from the department:
%% problem 2a, the second line of the third equation isn't doing anything.
%% problem 3a, you reference assignment 4. i think you mean problem 1
%% problem 3c, a good observation from zieba: the two different scaling regimes accounts for n1 estimate being too high, while
%% n2 and n3 too low, perhaps?

%% delete MAD question to 
%% give them a form to aim for

%% add question for making tree maps
%% code, matlab
%% produce little videos?
%%


%% For Q1--3, you'll explore random walks,
%% touching the Central Limit Theorem and
%% the first return problem.
%% There and back again.


%% \pleasesubmitprojectdraft

%% \assignmentsonly{\pleasesubmitprojectdraft}

Notes for Baby Name analysis:

\begin{itemize}
\item
  You will have the data sets on hand from the previous assignment.
  %%  \item 
  %%    Matlab. Yes. You will need to install it. Please connect with Assistant to the Regional Deliverator.
\item 
  Unix systems will work (Linux, the Apple things, etc.). Windows may not (alternative given below).
\item 
  As is, you will need the command epstopdf. Please install if not on deck already.
\end{itemize}


\begin{enumerate}

\item

  Plot time series for the rank
  of these baby names in the US over all years in the census data:

  \begin{itemize}
  \item 
    Shirley.
  \item
    Desmond.
  \item
    Madison.
  \item
    Aiden.
  \item
    A name of your choice.
  \end{itemize}

  Note that if you plotted relative frequency rather than rank,
  you would need to know (or estimate) the overall number of babies born.
  Ranks are both easy simple to work with and easy to understand.

%% Not using frequency 
%%  For each name, you will need to the frequency for each year, as well
  %%  as that year's total number of names (girls or boys).

  
   \solutionstart

   %% solution goes here

   \solutionend

\item (3 + 3 + 3 + 3)

  Only this question requires The Laboratory of the Matrix (Matlab).
  See alternative below if you cannot get the allotaxonometer to work on your system.

  Generate allotaxonographs comparing the following four pairs:

  \begin{enumerate}
  \item
    Baby girl names in 1952 versus baby girl names in 2002.
  \item
    Baby boy names in 1952 versus baby boy names in 2002.
  \item
    Baby girl names in 1952 versus baby boy names in 1952.
  \item
    Baby girl names in 2002 versus baby boy names in 2002.
  \end{enumerate}

  Use rank-turbulence divergence with $\alpha = 0$ and $\alpha = \infty$.

  Online appendices for main papers is
  here:
  \href{http://compstorylab.org/allotaxonometry/}{http://compstorylab.org/allotaxonometry/}.

  The gitlab repository:

  \url{https://gitlab.com/compstorylab/allotaxonometer/}

  For example baby names code, look through the main script here:
  
  \url{https://gitlab.com/compstorylab/allotaxonometer/figures/babynames/figures/}

  See if you can get this script to run as is.

  Contains overview, examples, links to papers, figure-making code, etc.

  
   \solutionstart

   %% solution goes here

   \solutionend

  \textbf{Alternative:}

  Using rank-turbulence divergence with $\alpha = 0$ and $\infty$,
  list the top 30 contributing baby names for the four comparisons listed above.
  
  Indicate which year each contributing baby name comes from in parentheses.

  For ordering, you do not need to compute RTD in full but rather just the core structure:
  \begin{equation}
  \left\lvert
  \frac{1}{\left[\zipfrank_{\elementsymbol,\indexaraw}\right]^{\alpha}}
  -
  \frac{1}{\left[\zipfrank_{\elementsymbol,\indexbraw}\right]^{\alpha}}
  \right\lvert.
  \label{eq:pocs-asst05.allotax-core}
  \end{equation}
  Recall that for $\alpha=0$ and $\alpha=\infty$, the essential core structure becomes:
  \begin{equation}
    \left\lvert
    \ln
    \frac{\zipfrank_{\elementsymbol,\indexaraw}
    }{\zipfrank_{\elementsymbol,\indexbraw}}
    \right\rvert
    \
    \mbox{and}
    \
    \max_{\elementsymbol}
    \left\{
    \frac{
      1
    }{
      \zipfrank_{\elementsymbol,\indexaraw}
    },
    \frac{
      1
    }{
      \zipfrank_{\elementsymbol,\indexbraw}
    }
    \right\}.
    \label{eq:pocs-asst05.allotax-core-alpha-0-and-infty}
  \end{equation}

  
   \solutionstart

   %% solution goes here

   \solutionend

\item 
  Everyday random walks and the Central Limit Theorem:

  Show that the observation
  that the number of discrete random walks of duration
  $t=2n$ starting at $x_{0}=0$ and ending at displacement $x_{2n} = 2k$ where
  $k \in \{0, \pm 1, \pm 2, \ldots, \pm n\}$
  is
  $$
  N(0,2k,2n)
  =
  \binom{2n}{n+k}
  =
  \binom{2n}{n-k}
  $$
  leads to a Gaussian distribution for large $t=2n$:
  $$
  \mathbf{Pr}(x_{t} \equiv x) 
  \simeq
  \frac{1}{\sqrt{2\pi t}}
  e^{-\frac{x^2}{2t}}.
  $$
  Please note that $k \ll n$.

  \wordwikilink{http://en.wikipedia.org/wiki/Stirling's_approximation}{Stirling's sterling approximation}
  will prove most helpful.

  \textbf{Hint:}
  You should be able to reach this form:
  $$
  \frac{
    \mbox{Some stuff not involving penguins}
  }{
    \mbox{Some other penguin-free stuff} \times
    (1-k^2/n^2)^{n+1/2} 
    (1+k/n)^{k}(1-k/n)^{-k}
  }.
  $$
  Lots of sneakiness here.  You'll want to examine
  the natural log of the piece shown above, and see
  how it behaves for large $n$.  

  You may very well need to use
  the Taylor expansion $\ln(1+z) \simeq z$.
  
  Exponentiate and carry on.

  \textbf{Tip:}
  If at any point penguins appear in your expression, 
  you're in real trouble.  Get some fresh air and start again.

  
   \solutionstart

   %% solution goes here

   \solutionend



\item 

  From lectures, show that the number of distinct 
  1-$d$ random walk that start at $x=i$ and end 
  at $x=j$ after $t$ time steps is
  $$
  N(i,j,t) = \binom{t}{(t+j-i)/2}. 
  $$
  Assume that $j$ is reachable from $i$ after $t$ time steps.

  \videohint{daSIYz-0U3E}{Counting random walks}

  %% Video hint: \url{http://www.youtube.com/watch?v=daSIYz-0U3E}.

  
   \solutionstart

   %% solution goes here

   \solutionend

\item 

  \textit{Discrete random walks:}

  In class, we argued that
  the number of random walks returning
  to the origin for the first time
  after $2n$ time steps is given by
  $$
  N_{\rm first\ return}(2n) 
  =
  N_{\rm fr}(2n) 
  = 
  N(1,1,2n-2) - N(-1,1,2n-2)
  $$
  where
  $$ 
  N(i,j,t) = \binom{t}{(t+j-i)/2}. 
  $$

  Find the leading order term for $N_{\rm fr}(2n)$
  as $n \rightarrow \infty$.

  Two-step approach:
  \begin{enumerate}
  \item 
    Combine the terms to form a single fraction,
  \item
    and then again use 
    \wordwikilink{http://en.wikipedia.org/wiki/Stirling's_approximation}{Stirling's bonza approximation}.
  \end{enumerate}

  If you enjoy this sort of thing, you may like to explore
  the same problem for random walks in higher dimensions.  
  Seek out George P\'{o}lya.

  And we are connecting to much other good stuff in combinatorics; more to 
  come in the solutions.

  %% \#toomuchexcitement

  
   \solutionstart

   %% solution goes here

   \solutionend
  


  


\end{enumerate}



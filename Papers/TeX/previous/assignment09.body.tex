\textbf{Name:} \\

\medskip

\textbf{Conspirators:} 

\medskip
\medskip

\hrule

\medskip


\assignmentsonly{\pleasesubmitprojectdraft}

Your assignment for this week will be to present
a plan for your project in a short video.\footnote{Please do not request to make a diorama instead of a video.
Because of the \censor{model railroad incident} in Professor \censor{Professorson}'s 2017 course on \censor{Conspiracy} \censor{Theories}, dioramas are now strictly forbidden.}

Details:

You will collectively generate a mini-conference of short talks,
with each PoCS team member presenting for two to three minutes
(max 3:00, 180 seconds).

Here's what you need to know and do:

First: If you have not already done so,
please firm up your team and project topic.

Per earlier guidance:
Feel free to talk in Teams about possible projects.
Pitch your idea and recruit people to your team.

Teams of 2 to 3 are strongly encouraged (4 is maybe too many, 1 is totally okay, 5 is right out).

Below are instructions for the talks and how to submit a video of your talk to Microsoft Streams
along with your slides within our Teams environment.

There should be one video and one set of slides per team.

Videos will be private to the course.

These talks always prove to be interesting, diverse, and fun.

Okay, here's the plan for these first talks:

\begin{enumerate}
\item
  Talks will be capped at 3:00 minutes per person.
\item
  Your mission, which you have accepted, is to:
  \begin{enumerate}
  \item 
    Clearly state the problem/area you're going to investigate;
  \item 
    Why it's interesting;\\
    and
  \item 
    What you plan to do for the remainder of the semester.
  \item 
    Please also quickly introduce yourself at the start (name + your field).
  \end{enumerate}

\item
  Talks should absolutely be PG and respectful of others.
\item
  If you are part of a group, you will need to speak for 3:00 minutes
  each. Please coordinate your talk with your fellow group members.
\item
  Talks that are longer than $n \times$3:00 will be removed and you will be asked to resubmit.
\item
  Slides: Mandatory. The number should be 1 to 3 per speaker.
  More
  can work but certainly not, say, 20, unless
  flipping through them rapidly helps with your presentation.
  Your assessment will in part be based on your slides.
\item
  Practice before recording!  These are short talks so you can run through
  them a number of times to straighten everything out.
\item
  Please submit your slides and video before the due date and time.
\item
  Talks will be organized in a Teams channel by the Assistant to the Regional Deliverator.
\item
  All students will be requested to watch all talks.
  Providing helpful comments and feedback via Teams is encouraged.
  
  %%     Naming convention:
  %%   nnCSYS300project-firsttalk-$firstname-$lastname-2018-10-18.pdf\\
  %%   where the leading
  %%   nn = your talk number, including a padded 0 if needed
  %% 
  %%   Examples:\\
  %%   07CSYS300project-firsttalk-michael-palin-2018-10-18.pdf\\
  %%   07CSYS300project-firsttalk-michael-palin-2018-10-18.pptx

\end{enumerate}



%% My machine will handle Powerpoint (it uses a pair of tongs
%% and rubber gloves to do so) but highly fancy Powerpoint presentations made
%% on a Windows machine may not transfer perfectly.  
%% If you are feeling up for Beamer/LaTeX, I highly encourage it.
%% Anything that ends up as a pdf should be fine.


\clearpage

A few random project ideas/topics/areas:
\begin{enumerate}
\item
  Work with data from Storywrangler. Join Twitter time series with
  data from other areas (e.g., ecology, energy use, food, politics) to
  open up new ways to examine socio-anything-ological studies.
\item
  Scaling of the number of meanings of words with frequency of usage
\item
  Work on ousiometry, allotaxonometry figures in Python, Julia, and/or D3.
\item
  Excess deaths, the true toll, and the mismeasurement of death:

  Ongoing:

  \wordwikilink{https://www.economist.com/graphic-detail/coronavirus-excess-deaths-tracker}{https://www.economist.com/graphic-detail/coronavirus-excess-deaths-tracker}

  No longer being updated:
  
  \wordwikilink{https://www.nytimes.com/interactive/2020/05/05/us/coronavirus-death-toll-us.html}{https://www.nytimes.com/interactive/2020/05/05/us/coronavirus-death-toll-us.html}.

  \wordwikilink{https://www.nytimes.com/interactive/2021/01/14/us/covid-19-death-toll.html}{https://www.nytimes.com/interactive/2021/01/14/us/covid-19-death-toll.html}

\item
  Everything in IMDB.

  Start from series heat visualizations:
  \wordwikilink{https://vallandingham.me/seriesheat/}{https://vallandingham.me/seriesheat/}

  Box office, ratings.

\item
  Everything to do with TV tropes:
  \wordwikilink{https://tvtropes.org}{https://tvtropes.org}

\item
  
  \wordwikilink{https://litlab.stanford.edu/pamphlets/}{https://litlab.stanford.edu/pamphlets/}

\item

  For more, see projects slides.

\end{enumerate}


%% \begin{enumerate}
%% \item
%%   Work with data from Storywrangler. Join Twitter time series with
%%   data from other areas (e.g., ecology, energy use, food, politics) to
%%   open up new ways to examine socio-anything-ological studies.
%% \item
%%   COVID-19 Pandemic
%% \item
%%   Black Lives Matter movement and protests in 2020 (see~\ ci te {gallagher2018a}).
%% \item
%%   Excess deaths, the true toll, and the mismeasurement of death:
%%   \wordwikilink{https://www.nytimes.com/interactive/2020/05/05/us/coronavirus-death-toll-us.html}{https://www.nytimes.com/interactive/2020/05/05/us/coronavirus-death-toll-us.html}.
%% \end{enumerate}
